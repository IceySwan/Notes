\chapter{Reverse Time Space NLS}
This chapter mainly introduces three inverse problems of nonlocal NLS, mainly referring to Yang's article\cite{YANG2019328}
\section{The coupled Schrödinger equations}
Consider the rverse-space NLS equation
\begin{equation}
    \rmi q_{t}(x,t) + q_{xx}(x,t) + 2 q^{2}(x,t) q^{*}(-x,t) = 0 \label{eq:reverse-x-NLS}
\end{equation}
reverse-time NLS equation
\begin{equation}
    \rmi q_{t}(x,t) + q_{xx}(x,t) + 2 q^{2}(x,t) q^{*}(x,-t) = 0 \label{eq:reverse-t-NLS}
\end{equation}
and the reverse-space-time NLS equation
\begin{equation}
    \rmi q_{t}(x,t) + q_{xx}(x,t) + 2 q^{2}(x,t) q^{*}(-x,-t) = 0 \label{eq:reverse-x-time-NLS}
\end{equation}
This equations can be derived form the following member of the AKNS hierarchy - the coupled NLS equations
\begin{equation}
    \rmi q_{t} + q_{xx} - 2q^{2}r = 0, \quad \rmi r_{t} - r_{xx} - 2r^{2}q = 0. \label{eq:coupled-NLS}
\end{equation}
Under reductions
\begin{align}
    r(x,t) &= -q^{*}(-x,t), \label{eq:reverse-x-NLS-reduction}\\
    r(x,t) &= -q(x,-t), \label{eq:reverse-t-NLS-reduction}\\
    r(x,t) &= -q(-x,-t), \label{eq:reverse-x-time-NLS-reduction}
\end{align}
these coupled equations reduce to the reverse-space NLS equation \eqref{eq:reverse-x-NLS}, the reverse-time NLS equation \eqref{eq:reverse-t-NLS} and the reverse-space-time NLS equation \eqref{eq:reverse-x-time-NLS} respectively.
\section{Symetry relations of scattering data in the nonlocal NLS equations}
We first present symmetry relations of the secattering data for the reverse-space NLS equation \eqref{eq:reverse-x-NLS} and the reverse-time NLS equation \eqref{eq:reverse-t-NLS}. The symmetry relations of the scattering data for the reverse-space-time NLS equation \eqref{eq:reverse-x-time-NLS} can be obtained in a similar way. For this pourpose, we first introduce some notations. We define
\begin{equation}
    \Lambda = \begin{pmatrix}
        1& 0 \\
        0 & -1
    \end{pmatrix}, \quad 
    \sigma_{1} = \begin{pmatrix}
        0 & 1 \\
        1 & 0
    \end{pmatrix}
\end{equation}
which is a Paili spin matrix, and use the
\subsection{The reverse-space NLS equation}
\begin{theorem}
    For the reverser-space NLS equation \eqref{eq:reverse-x-NLS}, if $ \zeta $ is an eigenvalue, so is $ -\zeta^{*} $. Thus, non-pirely-imaginary eigenvalues appear as pairs $(\zeta, -\zeta^{*})$, which lie in the same half of the complex plane. Symmetry ralations on the eigenvactors are given as follows: 
    \begin{enumerate}
        \item If $ (\zeta_{k}, \hat{\zeta_{k}}) \in \mathbb{C_{+}}$, then $\hat{\zeta_{k}} = - \zeta^{*}_{k} $, their column eigenvectors are related as $\hat{\mathbf{v}}_{k0} = \sigma_{1} \mathbf{v}_{k0}^{*}$.
        \item If $ \zeta_{k} \in i\mathbb{R}_{+} $, its eigenvectors is of the form $\mathbf{v}_{k0} = \begin{bmatrix}1 & e^{\rmi\theta_{k}}\end{bmatrix}^{T} $, where $ \theta_{k} $ is a real constant.
        \item If $ (\bar{\zeta_{k}}, \hat{\bar{\zeta_{k}}}) \in \mathbb{C_{-}} $, then $\hat{\bar{\zeta_{k}}} = - \bar{\zeta}^{*}_{k} $, their row eigenvectors are related as $\hat{\bar{\mathbf{v}}}_{k0} = \bar{\mathbf{v}}_{k0}^{*} \sigma_{1}$.
        \item If $ \bar{\zeta_{k}} \in i\mathbb{R}_{-} $, its eigenvectors is of the form $\bar{\mathbf{v}}_{k0} = \begin{bmatrix}1 & e^{\rmi\bar{\theta}_{k}}\end{bmatrix} $, where $ \bar{\theta}_{k} $ is a real constant.
    \end{enumerate}
\end{theorem}
To proof these results in perspective, we recall that for the local NLS equation, 
\begin{equation}
    \rmi q_{t} + q_{xx} \pm 2 q^{2} q^{*} = 0 \label{eq:local-NLS}
\end{equation}
which is obatined from the coupled Schrödinger equations \eqref{eq:coupled-NLS} under the reduction of $ r(x,t) = -q^{*}(x,t) $, the symmetry of its scattering data are $ \bar{\zeta_{k}} = -\zeta^{*}_{k} $ and $\bar{\mathbf{v}}_{k0} = \mathbf{v}_{k0}^{*} $. 

Thus, symmetry relations for the nonlocal NLS equations are different from those local NLS equations.  In particular, for the reverse-space and reverse-space-time NLS equations, eigenvalues in the upper and lower halves of the complex plane are completely independent. This independence allows for novel eigenvalue con-figurations, which will give rise to new types of multi-solitons. This will be demonstrated in the next section. 

Before proving this theorem, we first establish a connection between the discrete scattering datda for N-solitons $\{\zeta_{k}, \bar{\zeta_{k}}, a_{k}, b_{k}, \bar{a_{k}}, \bar{b_{k}}\}(1\neq k \neq N) $ and discrete eigenmodes in the eigenvalue problem $ Y_{x} = MY $ and its adjoint problem $ K_{x} = -KM $, where we set 
\begin{align}
    Y_{x} = - \rmi \zeta \Lambda Y + QY, \label{eq:reverse-x-NLS-eigenvalue}\\
    K_{x} = \rmi \zeta \Lambda K + KQ \label{eq:reverse-x-NLS-adjoint-eigenvalue}
\end{align}
where the potenitial matrix $ Q $ is given by
\begin{equation}
    Q = \begin{pmatrix}
        0 & q(x,0) \\
        r(x,0) & 0
    \end{pmatrix}
\end{equation} 
and $ q(x,0), r(x,0) $ are the initial conditions of functions $ q(x,t), r(x,t) $ at $ t = 0 $. Indeed $ \forall \{ \zeta_{k}, a_{k}, b_{k} \} $ of the discrete scattering data, where $ \zeta \in \mathbb{C_{+}} $ is the eigenvalue of \eqref{eq:reverse-x-NLS-eigenvalue}, whose discrete eigenfunction $ Y_{k} $ has the following asymptotics 
\begin{equation}
    Y_{k}(x) \to \begin{bmatrix} a_{k} \rme^{-\rmi \zeta_{k} x} \\ 0 \end{bmatrix}, \quad x \to -\infty, \quad Y_{k}(x) \to \begin{bmatrix} 0 \\ -b_{k} \rme^{\rmi \zeta_{k} x} \end{bmatrix}, \quad x \to +\infty
\end{equation}
Analogously, for the eigenvalue $ \bar{\zeta} \in \mathbb{C_{-}} $ of the adjoint eigenvalue problem \eqref{eq:reverse-x-NLS-adjoint-eigenvalue}, the discrete eigenfunction $ K_{k} $ has the following asymptotics 
\begin{equation}
    K_{k}(x) \to \begin{bmatrix} \bar{a}_{k} \rme^{-\rmi \bar{\zeta}_{k} x} & 0 \end{bmatrix}, \quad x \to -\infty, \quad K_{k}(x) \to \begin{bmatrix} 0 & -\bar{b}_{k}\rme^{-\rmi \bar{\zeta}_{k}x} \end{bmatrix}, \quad x \to +\infty
\end{equation}
In view of this connection, in order to derive symmetry relatiuons on the (discrete) scattering data
\subsection{The reverse-time NLS equation}
\subsection{The reverse-space-time NLS equation}