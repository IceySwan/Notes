\chapter{Preface}
\section*{Contents}
本笔记旨在分享笔者对 Riemann-Hilbert 方法的理解, RH 方法作为一种强大的数学工具, 在非线性偏微分方程的求解, 尤其是可积系统中具有重要的应用价值. 通过对其理论背景, 基本方法及具体应用的学习, 笔者希望为读者提供一个清晰的入门指引。

%本笔记主要参考复旦大学范恩贵老师的讲义\cite{Fan-RH}, 感谢郑州大学魏姣讨论班的支持.
主要内容如下: 

第一章主要介绍 RH 方法的背景知识, 包括 Plemelj 定理, RH 问题等. 

第二章主要介绍利用 RH 方法求解零边界的 NLS 方程, 通过构造特征函数, 分析其解析性与对称性, 以及建立相关的 RH 问题,最终推导出 NLS 方程的 N 孤子解. 本章节主要参考了复旦大学范恩贵老师的讲义\cite{Fan-RH}. 这一章节感谢我的同门 韩刻蓉 的帮助与讲解. 

第三章主要介绍利用 RH 方法求解反时间, 反空间, 及反时空的反演, 这些方程是耦合 NLS 方程在不同约束条件下的特殊形式。通过分析其散射数据的对称性,进一步推导出这些非局部方程的 N 孤子解. 本章节主要参考了杨建科老师的论文\cite{YANG2019328}

第四章为在第三章的基础上, 进一步推广到三维的反时空 NLS 方程. 通过分析其散射数据的对称性,推导出三维反时空 NLS 方程的 N 孤子解. 本章节构成了我的第一篇论文.
%\section*{符号约定}
%如无特别声明, 我们约定如下符号
%
%\begin{enumerate}
%    \item $ z^{*} $ 为 $ z $ 的共轭, 
%    \item $ A^{\dagger} $ 为 $ A $ 的 Hermitian 转置
%    \item $ A:= BCD $ 为将 $ BCD $ 记为 $ A $
%    \item $ \Leftarrow $ 表示利用括号内的东西, 推出前面结论如
%    \begin{equation}
%        \lim_{z \to z_{1}} \frac{s_{11}(z) - s_{11}(z_{1})}{z - z_{1}} (z - z^{*}) \Leftarrow (s_{11}(z) = 0)
%    \end{equation}
%    表示利用 $ s_{11}(z) = 0 $ 的事实, 可以得到 $ \lim_{z \to z_{1}} \frac{s_{11}(z) - s_{11}(z_{1})}{z - z_{1}} (z - z^{*}) $. 这里使用 $ \Leftarrow $ 表示, 以和 $ \implies/\Rightarrow $ 作区分. 
%    \item 为方便, 目前使用了括号 $ (a,b) $ 表示行向量 $ \begin{pmatrix} a & b\end{pmatrix} $ 后续或许会更改
%\end{enumerate}

\section*{感想}

本笔记为在老家过年期间编写, 整个过程并不算顺利, 老家的房子年久失修, 塌了外墙和两间屋顶, 白天需要和父亲一起修房子, 应付家庭琐事; 只有夜间才能缓慢的推进写作. 另外农村没有暖气, 零下十余度的环境实难称舒适, 需要字面意义上的争分夺秒. 尤其深夜伏案时, 恍若独行于漫长甬道, 徘徊在黑暗迷宫之中. 然正如 A. Zee 所言, 夜航人自有夜行法\cite{NightPhysics}. 历时一月的艰辛写作中, 笔者感到一种难以言喻的, 某种漠然的相互理解, 如越过高墙, 漫步在满月下的林地, 并在夜色中获得慰籍. 是邪,非邪?解释或属妄诞,感受毕竟真实. 

因仅为一家之言, 仓促间完成又后期无他人审校, 难免存在疏漏, 错误和挂一漏万之处. 望各位读者在阅读时能指出不足, 共同探讨与完善. 

序曲将终,敬无穷的远方,与无尽的人们.  


本笔记存档于 \href{https://github.com/IceySwan/Notes}{https://github.com/IceySwan/Notes}, 如发现任何错误请提 \href{https://github.com/IceySwan/Notes/issues/new}{issue} 或联系 \href{mailto:hi@icey.one}{hi@icey.one}

\begin{flushright}
    Icey Swan\\
    2025 年 2 月 %12于郑州大学
\end{flushright}