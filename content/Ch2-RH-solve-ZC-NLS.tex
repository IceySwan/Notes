\chapter{RH 方法求解零边界的 NLS 方程}
\section{聚焦 NLS 方程}
\subsection{特征函数}

考虑聚焦 NLS 方程的初值问题 
\begin{align}
\rmi q_{t}(x,t) + q_{xx}(x,t) + 2q(x,t)|q(x,t)|^{2} = 0 \label{eq:Focus-NLS}\\
q(x,0) = q_{0}(x) \in \mathcal{S}(\mathbb{R}) 
\end{align}
其中 $ \mathcal{S}(\mathbb{R}) $ 表示 Schwartz 空间. 
%\begin{equation}
%  \mathcal{S}(\mathbb{R}) = \left\{ f \in C^{\infty}(\mathbb{R}) : \| f \|_{\alpha, \beta} = \sum_{x \in \mathbb{R}} | x^{\alpha} \partial_{\beta}f(x) | < \infty, \alpha, \beta \in \mathbb{Z} \right\}
%\end{equation}
NLS 方程 (\ref{eq:Focus-NLS}) 具有如下矩阵形式的 Lax 对 
\begin{align}
  \psi_{x} + \rmi z \sigma_{3} \psi = P \psi \label{eq:Focus-NLS-Lax-x}\\
  \psi_{t} + 2 \rmi z^2 \sigma_{3} \psi = Q \psi \label{eq:Focus-NLS-Lax-t}
\end{align}
其中 
\begin{equation*}
  \sigma_{3} = \begin{pmatrix}
    1 & \\ & -1
  \end{pmatrix}, \quad
  P = \begin{pmatrix}
    & q \\ -q^{*} & 
  \end{pmatrix}, \quad
  Q = \begin{pmatrix}
    \rmi |q|^2 & \rmi q_{x} \\ \rmi q_{x}^{*} & -\rmi |q|^2
  \end{pmatrix} + 2 z P
\end{equation*}
满足零曲率方程
\begin{equation}
  \left(P - \rmi z \sigma_{3}\right)_{t} - \left(Q - 2 \rmi z^2 \sigma_{3}\right)_{x} + \left[P - \rmi z \sigma_{3}, Q - 2 \rmi z^2 \sigma_{3} \right] = 0
\end{equation}

\subsection{渐进性}

由于 $ q_{0}(x) \in \mathcal{S}(\mathbb{R}) $, 且 $ q(x,t), q_{x}(x,t) \to 0, (x \to \infty) $, 故 x 足够大时, $ P, Q $ 可忽略, 从而 Lax 对 (\ref{eq:Focus-NLS-Lax-x}) - (\ref{eq:Focus-NLS-Lax-t}) 可近似为
\begin{equation*}
  \psi_{x} \sim - \rmi z \sigma_{3} \psi, \quad \psi_{t} \sim - 2\rmi z^2 \sigma_{3} \psi
\end{equation*}
可得渐进形式的 Jost 解 $ \psi = e^{- \rmi \theta(z) \sigma_{3}}, (x \to \infty) $, 其中 $ \theta(z) = z x + 2z^2 t $. 下面为表示方便, 不妨记 $ E := e^{ \rmi \theta(z) \sigma_{3}} $, 做变换 $ \mu(x,t,z) = \psi(x,t,z)E $, 则有
\begin{equation}
  % 这里有问题!!!
  E^{-1} = e^{-\rmi \theta(z) \sigma_{3}}, \quad \mu(x,t,z) \to I( x \to \infty) \label{eq:mu-asymptomatic}
\end{equation}
则可得 
\begin{align}
  \mu_{x} = \psi_{x}E + \psi E (\rmi \sigma_{3} z) \\
  \mu_{t} = \psi_{t}E + \psi E (2 \rmi \sigma_{3} z^2)
\end{align}
带入 Lax 对 (\ref{eq:Focus-NLS-Lax-x}) - (\ref{eq:Focus-NLS-Lax-t}) 可得
\begin{align}
  \mu_{x} = (P \psi - \rmi z \sigma_{3} \psi)E + \psi E(\rmi \sigma_{3} z) \\
  \mu_{t} = (Q \psi - 2 \rmi z^2 \sigma_{3} \psi)E + \psi E (2 \rmi \hat{\sigma}_{3} z^2)
\end{align}
故有 
\begin{equation}
  \mu_{x} = P\mu -\rmi z [\sigma_{3}, \mu] \implies \mu_{x} + \rmi z [\sigma_{3}, \mu] = P \mu \label{eq:Focus-NLS-Lax-x-Asymptotic}
\end{equation}
同理可得
\begin{equation}
  \mu_{t} + 2 \rmi z^2 [\sigma_{3}, \mu] = Q \mu \label{eq:Focus-NLS-Lax-t-Asymptotic}
\end{equation}
由 (\ref{eq:Focus-NLS-Lax-x-Asymptotic}) 式可得
\begin{align}
  \mu_{x} + \rmi z \hat{\sigma} \mu &= P \mu \\
  e^{\rmi \theta(z) \hat{\sigma_{3}}} \mu_{x} + \rmi z \hat{\sigma}_{3} e^{\rmi \theta(z) \hat{\sigma_{3}}} \mu &= e^{\rmi \theta(z) \hat{\sigma}_{3}} P
\end{align}
其中 $ \hat{\sigma}_{3}X = [\sigma_{3}, X], e^{\rmi \theta(z) \hat{\sigma}_{3}} X = E X E^{-1} $, 为表示方便不妨记 $ \hat{E} = e^{\rmi \theta(z) \hat{\sigma}_{3}} $. 
易得 Lax 对的全微分形式
\begin{equation}
  \rmd (\hat{E} \mu) = \hat{E} [ (P \rmd x + Q \rmd t) \mu ] \label{eq:diffision-mu}
\end{equation}
为构造规范的 Riemann-Hilbert 问题, 即其解在 $ z \to \infty $ 时渐进于单位阵, 现只需证明 $ \mu \to I (z \to \infty) $. 

\begin{proof}
  将 $ \mu $ 在无穷远点 Taylor 展开
  \begin{equation}
    \mu = \mu^{(0)} + \frac{1}{z} \mu^{(1)} + \cdots = \mu^{(0)} + O(z^{-1}) \label{eq:mu-Taylor}
  \end{equation}
  其中 $ \mu^{(0)}, \mu^{(1)} $ 与 $ z $ 无关, 将上式代入 (\ref{eq:Focus-NLS-Lax-x-Asymptotic}), (\ref{eq:Focus-NLS-Lax-t-Asymptotic}) 比较 $ z $ 的次数可得
  \begin{equation}
    \begin{aligned}
      \left[\mu^{(0)} + O(z^{-1})\right]_{x} + \rmi z \left[\sigma_{3}, \mu^{(0)} + \frac{\mu^{(1)}}{z} + O(z^{-2})\right] &= P\left[\mu^{(0)} + O(z^{-1})\right] \\
      [\mu^{(0)} + O(z^{-1})]_{t} + 2 \rmi z^2 [\sigma_{3}, \mu^{(0)} + \frac{\mu^{(1)}}{z} + O(z^{-2})] &= (Q_{0} + 2 z P)[\mu^{(0)} + O(z^{-1})]
    \end{aligned}
  \end{equation}
  可得 $ x $ 部分: 
  \begin{equation}
    \begin{aligned}
      z:& \quad[\sigma_{3}, \mu^{(0)}] = 0 \\
      z^{0}:& \quad \mu^{(0)} + 2 \rmi z [\sigma_{3}, \mu^{(1)}] = P \mu^{(0)}
    \end{aligned}
    \implies \mu^{(0)} \text{为对角阵}
  \end{equation}
  $ t $ 部分: 
  \begin{equation}
    \begin{aligned}
      z^{2}:& \quad [\sigma_{3}, \mu^{(0)}] = 0 \\
      z:& \quad \rmi [\sigma_{3}, \mu^{(1)}] =  P \mu^{(1)} \label{eq:mu-Taylor-t-z}
    \end{aligned}
    \implies \mu^{(0)}_x = 0
  \end{equation}
  故 $ \mu^{(0)} $ 为与 $ x $ 无关的对角矩阵, 因此 (\ref{eq:mu-Taylor}) 式对 $ x, z $ 同时取极限, 并交换极限顺序可得 
  \begin{equation}
    \lim_{z \to \infty} \lim_{x \to \infty} \mu = \lim_{x \to \infty} \lim_{z \to \infty} \left( \mu^{(0)} + O(z^{(-1)}) \right)
  \end{equation}
  利用 (\ref{eq:mu-asymptomatic}), (\ref{eq:mu-Taylor}) 式可得 $ \mu^{(0)} = I $, 故 $ \mu \to I (z \to \infty) $.
\end{proof}
再将 $ \mu^{(0)} = I $ 带入 (\ref{eq:mu-Taylor-t-z}) 式比较矩阵对角元素得
\begin{equation}
  q(x,t) = 2 \rmi (\mu^{(1)})_{12} = 2 \rmi \lim_{z \to \infty} (z\mu)_{12}
\end{equation}
\begin{remark}
    上式可将 NLS 方程与特征函数联系起来, 接下来将特征函数与 RH 问题建立联系. 从而 NLS 方程的解可用 RH 问题的解表示, 然后通过 RH 问题反接触 NLS 方程的解, 即
    \begin{equation*}
      NLS \rightleftharpoons \text{特征函数} \rightleftharpoons RH \text{问题}
    \end{equation*}
\end{remark}

\section{解析性与对称性}\label{sec:analytic-symmetry}
由于 (\ref{eq:diffision-mu}) 为全微分形式, 积分与路径无关, 故选择两个特殊路径: 
\begin{equation*}
  (- \infty, t) \to(x,t) , \quad (+ \infty, t) \to(x,t)
\end{equation*}
因此可获得 Lax 对 (\ref{eq:Focus-NLS-Lax-x-Asymptotic}), (\ref{eq:Focus-NLS-Lax-t-Asymptotic}) 的两个特征函数 
\begin{equation}
  \mu_{1} = I - \int^{x}_{-\infty} e^{- \rmi z (x - y) \sigma_{3}} P \mu_{1}\rmd x, \quad \mu_{2} = I - \int^{x}_{+\infty} e^{- \rmi z (x - y) \sigma_{3}} P \mu_{2}\rmd x \label{eq:integral-mu}
\end{equation}
\begin{proof}
  由 (\ref{eq:diffision-mu}) 式, 由 $ (\hat{E} \mu)_{x} = \hat{E} P \mu $ 可得
  \begin{equation}
    \hat{E} \mu |_{-\infty}^{x} = \int_{-\infty}^{x} \hat{E} P \mu \rmd y
  \end{equation}
  \begin{equation*}
      LHS = \hat{E} \mu - \hat{E} I , \quad
      RHS = \int_{-\infty}^{x} e^{ \rmi zy + z^2 t \hat{\sigma}_{3}} P \mu \rmd y 
  \end{equation*}两边同时乘以 $ \hat{E} $, 即得 $ \mu_{1} $, $ \mu_{2} $ 同理可得. 
\end{proof}

显然, 其仍具有如下性质
\begin{equation}
  \mu_{1}, \mu_{2} \to I (x \to \pm \infty), \quad  \mu_{1}, \mu_{2} \to I (z \to + \infty)
\end{equation}
由于 $ \psi_{1} = \mu_{1} E^{-1}, \psi_{2} = \mu_{2} E^{-1} $, 为 Lax 对的两个解, 而 Lax 对为一节齐次线性方程组, 故这两个解线性相关, 故有
\begin{equation}
  \mu_{1}(x,t,z) = \mu_{2}(x,t,z) \hat{E} S(z), \label{eq:eig-mu-relationship} \quad 
  S(z) = 
  \begin{pmatrix}
    S_{11}(z) & S_{12}(z) \\ S_{21}(z) & S_{22}(z)
  \end{pmatrix}
\end{equation}
其中矩阵 $ S(z) $ 与 $ x,t $ 无关, 称为谱函数矩阵. 

再由 $ \mu = \psi E $ 可知 $ \det(\mu_{j}) = \det(\psi_{j}) $. 又因为 $ tr(P - \rmi z \sigma_{3}) = tr(Q - 2 \rmi z^2 \sigma_{3}) = 0 $, 由 Abel 定理可得
\begin{equation}
  \det(\psi_{j})_{x} = \det(\psi_{j})_{t} = 0
\end{equation}
故有 $ \det(\mu_{j})_x = \det(\mu_{j})_{t} = 0 $, 这说明 $ \det(\mu_{j}) $ 与 $ x,t $ 无关, 再由渐进性可得
\begin{equation}
  \det(\mu_{j}) = \lim_{|x| \to \infty} \det(\mu_{j}) = 1, \quad(j = 1,2) \label{eq:mu-det}
\end{equation}
故对 (\ref{eq:eig-mu-relationship}) 两边取行列式有 $ \det S(z) = 1 $. 

\subsection{解析性}
下面我们考虑特征函数 $ \mu_{1}, \mu_{1} $ 和谱矩阵 $ S(z) $ 的解析性, 记 $ \mu_{1} $ 的第一二列分别为
\begin{equation}
  \mu_{1} =
  \begin{pmatrix}
    \mu_{1}^{11} & \mu_{1}^{12} \\ \mu_{1}^{21} & \mu_{1}^{22}
  \end{pmatrix} = (\mu_{1}^{+}, \mu_{1}^{-}) \label{eq:mu-1-matrix}
\end{equation}
则由积分方程 (\ref{eq:integral-mu}) 可得如下 Voltarra 积分方程
  \begin{align}
    \mu_{1}^{+}(x,t,z) &= (1,0)^T - \int^{x}_{-\infty}  \begin{pmatrix} 1 &  \\  & e^{ \rmi z (x - y)} \end{pmatrix} P \mu_{1}^{+} \rmd y \label{eq:Voltarra-mu-1}\\
    \mu_{1}^{-}(x,t,z) &= (0,1)^T - \int^{x}_{-\infty}  \begin{pmatrix} e^{-\rmi z (x - y)} &  \\  & 1 \end{pmatrix} P \mu_{1}^{-} \rmd y \label{eq:Voltarra-mu-2}
  \end{align}
对于上面两方程, 由于积分变量 $ y < x $, 可得
\begin{equation*}
  e^{2 \rmi Z (x-y)} = e^{2 \rmi (x-y)\Re(z)}  e^{-2 \rmi (x-y)\Im(z)}, \quad   e^{- 2 \rmi Z (x-y)} = e^{- 2 \rmi (x-y)\Re(z)}  e^{2 \rmi (x-y)\Im(z)}  
\end{equation*}
因此当 $ q(x) \in L^{1}(\mathbb{R}) $ 时, 通过构造序列与 Neumann 级数, 可得 $ \mu_{1}^{\pm}, \mu_{1}^{-} $ 分别在 $ \mathbb{C}_{\pm} $ 解析性.

\begin{remark}
上面 Voltarra 积分的推导: 考虑到 $ e^{\hat{\sigma}_{3}} X = e^{\sigma_{3}} X e^{-\sigma_{3}} $ ,故有
\begin{equation*}
  \begin{aligned}
  \mu_{1} &= I + \int_{-\infty}^{x} e^{- \rmi z(x-y) \hat{\sigma}_{3}} P \mu_{1} \rmd y = I + \int_{-\infty}^{x} e^{- \rmi z(x-y) \sigma_{3}} P e^{\rmi z(x-y) \sigma_{3}} \mu_{1} \rmd y \\
          &= I + \int_{-\infty}^{x} \begin{pmatrix}
            & qe^{2\rmi z(x-y)} \\ - q^{*}e^{-2\rmi z(x-y)}  & 
          \end{pmatrix} \mu \rmd y
  \end{aligned} 
\end{equation*}
故 $ \mu_{1}^{+} = , \quad \mu_{1}^{-} = $
\end{remark}
\begin{proof}
  下证 $ \mu_{1}^{+} $ 的解析性. 

  Step1. 解的存在性: 事实上方程 (\ref{eq:Voltarra-mu-1}) 有如下 Neumann 级数分解
  \begin{equation}
    \mu_{1}^{+} = \sum_{n=0}^{\infty} c_{n}(x,z), \quad c_{n+1} = \int_{-\infty}^{x} \begin{pmatrix} 1 &  \\  & e^{ \rmi z (x - y)} \end{pmatrix} P c_{n}(y,z) \rmd y \label{eq:Neumann-series-mu-1+}
  \end{equation}
  其分量形式为 
  \begin{equation}
    c_{n+1}^{(1)}(x,z) = \int_{-\infty}^{x} q(y) c_{n}^{(2)}(y,z) \rmd y, \quad c_{n+1}^{(2)}(x,z) = \int_{-\infty}^{x} e^{2 \rmi z(x-y)} p(y) c_{n}^{(1)}(y,z) \rmd y
  \end{equation}
  其中  $c_{n+1} = (c_{n+1}^{(1)}, c_{n+1}^{(2)})^T, p = -q^{*} $. 
  由 $ c_{0}^{(2)} = 0 \implies c_{2n+1}^{(1)} = c_{2n}^{(2)} = 0 $, 故上面方程可简化为
  \begin{equation}
    c_{2n}^{(1)}(x,z) = \int_{-\infty}^{x} q(y) c_{2n-1}^{(2)}(y,z) \rmd y, \quad c_{2n+1}^{(2)}(x,z) = \int_{-\infty}^{x} e^{2 \rmi z(x-y)} p(y) c_{2n}^{(1)}(y,z) \rmd y \label{eq:Neumann-series-coefficient}
  \end{equation}
  对上面方程组进一步简化, 引入如下恒等式
  \begin{equation}
    \begin{aligned}
    &\frac{1}{j!} \int_{-\infty}^{x} | f(\xi) | \left[\int_{-\infty}^{\xi} | f(\eta) | \right]^{j} \rmd \xi = \frac{1}{(j+1)!} \int_{-\infty}^{x} \frac{d}{d \xi} \left[\int_{-\infty}^{\xi} | f(\eta) | \right]^{j+1} \rmd \xi \\
    = &\frac{1}{(j+1)!} \left[\int_{-\infty}^{x} | f(\xi) | d \xi \right]^{j+1} \quad (f \in L^{1}(\mathbb{R}))
    \end{aligned} \label{eq:Neumann-series-coefficient-identity}
  \end{equation}
  利用归纳法可证明当 $ \Im(z) > 0 $ 时, 有
  \begin{equation}
      |c_{2n+1}^{(2)} | = \frac{u^{n+1}(x)}{(n+1)!} \frac{v^{n}(x)}{n!} , \quad |c_{2n}^{(1)} | = \frac{u^{n}(x)}{n!} \frac{v^{n}(x)}{n!} \label{eq:Recurrence-relation}
  \end{equation}
  其中 
  \begin{equation}
      u(x) = \int_{-\infty}^{x} |p(y)| \rmd y, \quad v(x) = \int_{-\infty}^{x} |q(y)| \rmd y
  \end{equation}
  事实上, 当 $ \Im(z) > 0 $ 时, $ |e^{2 \rmi z (x-y)} | \leq 1 $, 利用 $ c_{0}^{(1)} = 1 $, 由 (\ref{eq:Recurrence-relation}) 可得
  \begin{equation}
    c_{1}^{(2)}(x,y) = \int_{-\infty}^{x} e^{2 \rmi z(x-y)} p(y) \rmd y, \quad c_{2}^{(1)}(x,y) = \int_{-\infty}^{x} e^{2 \rmi z(x-y)} q(y) c_{1}^{(2)} \rmd y
  \end{equation}
  又由 $ u(x) \geq 0, v(x) \geq 0 $, 有 $ c_{1}^{(2)} \leq v(x) $
  \begin{equation}
    \begin{aligned}
      |  c_{2}^{(1)}(x,y) &\leq \int_{-\infty}^{x} | q |  c_{1}^{(2)}(y,z) \rmd y \leq  \int_{-\infty}^{x} | q | v(y) \rmd y = \int_{-\infty}^{x} v'(y) u(y) \rmd y \\
      &= u(x) v(x) - \int_{-\infty}^{x} u'(y) v(y) \rmd y \leq u(x)v(x)
    \end{aligned}
  \end{equation}
  进而可得
  \begin{equation}
    \begin{aligned}
      | c_{3}^{(2)}(x,z) | &\leq \int_{-\infty}^{x} | p | c_{2}^{(1)}(y,z) \rmd y \leq \int_{-\infty}^{x} | p | u(y) v(y) \rmd y \\
      &\leq \int_{-\infty}^{x} u'(y) u(y) v(y) \rmd y = \frac{1}{2}u^2(x) v(x) - \int_{-\infty}^{x} \frac{1}{2}u^2(x) v'(x)  \rmd y \leq \frac{1}{2} u^{2}(x) v(x)
    \end{aligned}
  \end{equation}
  不妨设对于 $ k \leq l $ 时满足 $  | c_{2l+1}^{(2)} | \leq \frac{u^{l+1}(x)}{(l+1)!} \frac{v^{l}(x)}{l!}, | c_{2l}^{(1)} | \leq \frac{u^{l}(x)}{l!} \frac{v^{l}(x)}{l!} $, 下证对于 $ k = l +1 $ 仍成立, 利用 (\ref{eq:Neumann-series-coefficient-identity}) 式可得
    \begin{equation}
      | c_{2l+2}^{(1)} | \leq \int_{-\infty}^{x} | q(y) | \frac{u^{l+1}(y)}{(l+1)!} \frac{v^{l}(y)}{l!} \rmd y \leq \int_{-\infty}^{x} v'(y) \frac{u^{l+1}(y)}{(l+1)!} \frac{v^{l}(y)}{l!} \rmd y = \frac{u^{l+1}}{(l+1)!}  \frac{v^{l+1}}{(l+1)!} 
    \end{equation}
    及
    \begin{equation}
      | c_{2l+3}^{(2)} | \leq \int_{-\infty}^{x} | p(y) | \frac{u^{l+1}(y)}{(l+1)!} \frac{v^{l}(y)}{l!} \rmd y \leq \int_{-\infty}^{x} u'(y) \frac{u^{l+1}(y)}{(l+1)!} \frac{v^{l}(y)}{l!} \rmd y = \frac{u^{l+2}}{(l+2)!}  \frac{v^{l+1}}{(l+1)!}
    \end{equation}
  从而上述 (\ref{eq:Neumann-series-coefficient}) 式得证. 注意到 $ | q(x) |  = | p(x) |$, 则 $ v(x) = u(x) $, 故 (\ref{eq:Neumann-series-coefficient}) 式可化简为
  \begin{equation}
    | c_{2n+1}^{(2)} | \leq \frac{u^{2n+1}}{n!(n+1)!}, \quad | c_{2n}^{(1)} | \leq \frac{u^{2n}}{n!n!}
  \end{equation}
  当 $ q \in L^1(\mathbb{R}) $ 时, 有 $ u(x) $ 的上述级数均收敛, 故 Neumann 级数 $ \sum_{n=1}^{\infty} c_{n}(x,z) $ 绝对收敛, 此时 $ \mu_{1}^{+} $ 在 $ Im z > 0 $ 上解析, 且在 $ Im z \geq 0 $ 上连续.

  Step2. 解的唯一性: 不妨设 $ \tilde{\mu}_{1}^{+} $ 为方程 (\ref{eq:Voltarra-mu-1}) 的另一个解, 不妨设 $ h = \mu_{1}^{+} - \tilde{\mu}_{1}^{+} $, 则有 
  \begin{equation}
    \| h(x,t,z) \| = \left\| \int_{-\infty}^{x} \begin{pmatrix} 1 &  \\  & e^{ \rmi z (x - y)} \end{pmatrix} P h \rmd y \right\| \leq 2 \int_{-\infty}^{x} | q | \|  h \| \rmd y \implies \| h(x,t,z) \| \equiv  0
  \end{equation}
  故 $ \mu_{1}^{+} $ 为方程的唯一解, 同理可得 $ \mu_{1}^{-} $ 的解析性.
\end{proof}

所以可得 $ \mu_{1}^{+}, \mu_{1}^{-} $ 分别在 $ \mathbb{C}_{+}, \mathbb{C}_{-} $ 上解析. 同理可得 $ \mu_{2} $ 的第一二列分别在 $ \mathbb{C}_{-}, \mathbb{C}_{+} $ 上解析. 记作 
\begin{equation*}
 \mu_{2} = 
 \begin{pmatrix} 
  \mu_{2}^{(11)} & \mu_{2}^{(12)} \\ \mu_{2}^{(21)} & \mu_{2}^{(22)} 
\end{pmatrix} = (\mu_{2}^{-}, \mu_{2}^{+})
\end{equation*}
由 (\ref{eq:mu-det}) 可知, $ \mu_{1}, \mu_{2} $ 可逆, 且逆矩阵为其伴随阵, 另外基于 $ \mu_{1}, \mu_{2} $ 的列向量函数的解析性, 可得 $ \mu_{1}^{-1} $ 的第一二行在 $ \mathbb{C}_{-}, \mathbb{C}_{+} $ 上解析, $ \mu_{2}^{-1} $ 的第一二行在 $ \mathbb{C}_{+}, \mathbb{C}_{-} $ 上解析. 记为
\begin{equation}
  \mu_{1}^{-1} = 
  \begin{pmatrix} 
   \mu_{1}^{(22)} & - \mu_{1}^{(21)} \\ - \mu_{1}^{(12)} & \mu_{1}^{(11)} 
 \end{pmatrix} = \begin{pmatrix}
  - \hat{\mu_{1}^{-}} \\ \hat{\mu_{1}^{+}}
 \end{pmatrix} 
 , \quad 
  \mu_{2}^{-1} = 
  \begin{pmatrix} 
   \mu_{2}^{(22)} & - \mu_{2}^{(21)} \\ - \mu_{2}^{(12)} & \mu_{2}^{(11)} 
 \end{pmatrix} = \begin{pmatrix}
  - \hat{\mu_{2}^{+}} \\ \hat{\mu_{2}^{-}}
 \end{pmatrix} \label{eq:mu-1-2-inverse-matrix}
\end{equation}
利用 (\ref{eq:eig-mu-relationship}), (\ref{eq:mu-1-matrix}), (\ref{eq:mu-1-2-inverse-matrix}) 可得谱函数 $ S(z) $ 的解析性
\begin{equation}
  S(z) = \mu_{2}^{-1} \mu_{1} = \begin{pmatrix} - \hat{\mu_{2}^{+}} \\ \hat{\mu_{2}^{-}} \end{pmatrix} (\mu_{1}^{+}, \mu_{1}^{-}) 
  = \begin{pmatrix} - \hat{\mu_{2}^{+}} \mu_{1}^{+} & - \hat{\mu_{2}^{+}} \mu_{1}^{-} \\ \hat{\mu_{2}^{-}} \mu_{1}^{+} & \hat{\mu_{2}^{-}} \mu_{1}^{-} \end{pmatrix}
\end{equation}
可见 $ S_{11}(z) $ 在 $ \mathbb{C}_{+} $ 解析, $ S_{22}(z) $ 在 $ \mathbb{C}_{-} $ 解析, $ S_{12}, S_{21} $ 在 $ \mathbb{C}_{+}, \mathbb{C}_{-} $ 均不解析, 但连续到边界.

\subsection{对称性}
\begin{theorem}
  如上构造的特征函数 $ \mu_{1}, \mu_{2} $ 与谱函数 $ S(z) $ 具有如下对称性
  \begin{equation}
    \mu_{j}^{\dagger}(x,t,z^{*}) = \mu_{j}^{-1}(x,t,z), \quad S^{\dagger}(z^{*}) = S^{-1}(z) \quad( j = 1,2 )
  \end{equation}
\end{theorem}
\begin{proof}
  由 (\ref{eq:Focus-NLS-Lax-x-Asymptotic}) 有
  \begin{equation}
    \mu_{j,x}(x,t,z) = \rmi z [\sigma_{3}, \mu_{j}(x,t,z)] = P \mu_{j}(x,t,z) \label{eq:mu-diff-x}
  \end{equation}
  将 $ z $ 替换为 $ z^{*} $, 并在两边同时取 Hermite 共轭, 则有
  \begin{equation}
    \mu_{j,x}^{\dagger}(x,t,z^{*}) + \rmi z [\sigma_{3}, \mu_{j}^{\dagger}(X,t,z^{*})] = \mu_{j}^{\dagger}(x,t,z^{*}) P^{\dagger} 
  \end{equation}
  由于 $ P^{\dagger} = -P $, 故上式可化为
  \begin{equation}
    \mu_{j,x}^{\dagger}(x,t,z^{*}) + \rmi z [\sigma_{3}, \mu_{j}^{\dagger}(X,t,z^{*})] = - \mu_{j}^{\dagger}(x,t,z^{*}) P 
  \end{equation}
  另外对于 $ \mu_{j} \cdot \mu_{j}^{-1} = I $ 对 $ x $ 求偏导有
  \begin{equation}
    \mu_{j,x} \mu_{j}^{-1} + \mu_{j} \mu_{j,x}^{-1} = 0 \implies \mu_{j,x}^{-1}  = - \mu_{j}^{-1} \mu_{j,x} \mu_{j}^{-1}
  \end{equation}
  将 (\ref{eq:mu-diff-x}) 带入上式有 
  \begin{equation}
    \mu_{j,x}^{-1} = - \mu_{j}^{-1} ( P \mu_{j}(x,t,z) - \rmi z [\sigma_{3}, \mu_{j}(x,t,z)] \mu_{j}^{-1}(x,t,z) \implies \mu_{j,x}^{-1} + \rmi z [\sigma_{3},\mu_{j}^{-1}] = \mu_{j}^{-1} P
  \end{equation}
  由 (\ref{eq:Voltarra-mu-1}), (\ref{eq:Neumann-series-mu-1+}) 可得 $ \mu_{j}^{\dagger}, \mu_{j}^{-1} $ 满足相同的一次线性微分方程, 具有相同的渐进性, 又因为
  \begin{equation}
    \mu_{j}^{\dagger}(x,t,z^{*}) , \mu_{j}^{-1}(x,t,z) \to I , \quad | x | \to \infty \label{eq:mu-Hermitian-asymptotic}
  \end{equation}
  因此两者相等, 得到对称关系 $ \mu_{j}^{\dagger}(x,t,z^{*}) = \mu_{j}^{-1}(x,t,z), (j = 1,2) $. 接下来考虑 $ S $ 的对称性, 将 (\ref{eq:eig-mu-relationship}) 式改写为
  \begin{equation}
    S(z) = e^{\rmi \theta(z) \hat{\sigma}_{3}} (\mu_{2}^{-1}\mu_{1}) = \hat{E}(\mu_{2}^{-1}\mu_{1}) 
  \end{equation}
  由 (\ref{eq:mu-Hermitian-asymptotic}) 可得如下
  \begin{equation}
    S(z^{*})^{\dagger} = e^{\rmi \theta(z^{*}) \sigma_{3}} (\mu_{2}^{-1}\mu_{1})  e^{-\rmi \theta(z^{*})} = E \mu_{2}^{-1}\mu_{1} E^{-1} = E \mu_{1}^{\dagger} (\mu_{2}^{-1})^{\dagger} E^{-1} = E \mu_{1}^{\dagger} \mu_{2} E^{-1} = S(z)^{-1} 
  \end{equation}
  比较对应元素, 有
  \begin{equation}
    S_{11}^{*}(z^{*}) = S_{22}(z), S_{12}^{*}(z^{*}) = - S_{21}(z) \label{eq:S-symmetry}
  \end{equation} 
\end{proof}

\section{相关的 RH 问题}
\subsection{规范 RH 问题}
基于 \ref{sec:analytic-symmetry} 节的结论, 我们构造 RH 问题, 引入记号
\begin{equation}
  H_{1} = \begin{pmatrix}
    1 & \\ & 0
  \end{pmatrix}, \quad 
  H_{2} = \begin{pmatrix}
    0 & \\ & 1
  \end{pmatrix}
\end{equation}
并定义两个矩阵 
\begin{equation}
  \begin{aligned}
    P_{+}(x,t,z) = \mu_{1} H_{1} + \mu_{2} H_{2} = \begin{pmatrix}  \mu_{1}^{(11)} & \mu_{2}^{(12)} \\ \mu_{1}^{(21)} & \mu_{2}^{(22)} \end{pmatrix} = (\mu_{1}^{+}, \mu_{2}^{+}) \\
    P_{-}(x,t,z) =  H_{1} \mu_{1}^{-1} +  H_{2}\mu_{2}^{-1} = \begin{pmatrix}  \mu_{1}^{(22)} & -\mu_{1}^{(12)} \\ -\mu_{2}^{(21)} & \mu_{2}^{(11)} \end{pmatrix} = \begin{pmatrix} \hat{\mu}_{1}^{-} \\ \hat{\mu}_{2}^{-} \end{pmatrix}
  \end{aligned}
\end{equation}
则由 $ \mu_{j}, \mu_{j}^{-1} (j=1,2) $ 的解析性与渐进性, 直接可得 $ P_{+} $ 在 $ \mathbb{C}_{+} $ 上解析, $ P_{-} $ 在 $ \mathbb{C}_{-} $ 上解析, 且具有如下渐进性
\begin{equation}
  P_{+}, P_{-} \to I \quad | x | \to \infty
\end{equation}
由此可证明如下对称关系
\begin{theorem}
  $ P_{+}(x,t,z), P_{-}(x,t,z) $ 具有如下对称性
  \begin{equation}
    P_{+}^{\dagger}(x,t,z^{*}) =  P_{-}(x,t,z) \label{eq:P-symmetry}
  \end{equation}
\end{theorem}
\begin{proof}
  利用 (\ref{eq:mu-Hermitian-asymptotic}) 可得
  \begin{equation}
    \begin{aligned}
    P_{+}^{\dagger}(x,t,z^{*}) &= (\mu_{1}(x,t,z) H_{1} + \mu_{2}(x,t,z) H_{2} )^{\dagger} = H_{1} \mu_{1}^{\dagger}(z^{*}) + H_{2}\mu_{2}^{\dagger}(z^{*}) \\
      &= H_{1}\mu_{1}^{-1}(x,t,z) + H_{2} \mu_{2}^{-1}(x,t,z) = P_{-}(x,t,z)
    \end{aligned}
  \end{equation}
\end{proof}

\begin{definition}
  概括上面结果, 我们可以得到下面 RH 问题
  \begin{align}
    P_{\pm}(x,t,z) \in C(\mathbb{C}_{\pm})  \label{eq:RHP-1} \\
    p_{-}p_{+} = G(x,t,z) \quad z \in \mathbb{R} \label{eq:RHP-2}\\
    P_{\pm} \to I \quad z \to \infty \label{eq:RHP-3}
  \end{align}
  其中跳跃矩阵为
  \begin{equation}
    G(x,t,z) = \hat{E}^{-1} \begin{pmatrix}
      1 & -S_{12}(z) \\ S_{21}(z) & 1
    \end{pmatrix}
  \end{equation}
  进一步 NLS 方程的解 $ q(x,t) $ 可由 RH 问题的解给出
  \begin{equation}
    q(x,t) = 2 \rmi \lim_{z \to \infty} (z P_{+})_{12} = 2 \rmi (P_{+}^{(1)})_{12}
  \end{equation}
  其中 $ P_{+} = I + \frac{P_{+}^{(1)}}{z} + O(z^{-2}) $
\end{definition}
\begin{proof}
  只需证明跳跃关系即可, 注意到 $ \hat{E} H_{j} = H_{j}, (j=1,2) $ 可得 
  \begin{equation}
    \begin{aligned}
      G &= P_{-}P_{+} = (H_{1} \mu_{1}^{-1} + H_{2} \mu_{2}^{-1})(\mu_{1} H_{1} + \mu_{2} H_{2}) \\
        &\overset{(\ref{eq:eig-mu-relationship})}{=} [H_{1}(\hat{E}^{-1} S^{-1})\mu_{2}^{-1} + H_{2}\mu_{2}^{-1}]  [\mu_{2}(\hat{E}^{-1} S)H_{1} + \mu_{2}H_{2}] \\
        &= \hat{E}^{-1} [(H_{1}S^{-1} + H_2)(SH_{1} + H_{2})] = \hat{E}^{-1} \begin{pmatrix}
          1 & -S_{12}(z) \\ S_{21}(z) & 1
        \end{pmatrix}
    \end{aligned}
  \end{equation}
  直接计算可得
  \begin{equation}
    \begin{aligned}
    \det(P_{+}) &= \det(\mu_{1}H_{1} + \mu_{2} H_{2}) = \det(\mu_{2} \hat{E}^{-1} S) H_{1} + \mu_{2}H_{2}) \\
    &=\det(\mu_{2}) \det((\hat{E}^{-1} S) H_{1} + \mu_{2}H_{2}) = S_{11}(z)
    \end{aligned}
  \end{equation}
  同理, 有 $ \det(P_{-}) = S_{22}(z) $
\end{proof}

\subsection{RH 问题的可解性}
下面分两种情况讨论 RH 问题 (\ref{eq:RHP-1}) - (\ref{eq:RHP-3}) 的解

Case1. 如果 
\begin{equation}
  \det P_{\pm} (z) \neq 0 (\forall z \in \mathbb{C}), \label{eq:RHP-regular}
\end{equation} 
则称 RH 问题 (\ref{eq:RHP-1}) - (\ref{eq:RHP-3}) 为正则的, 将方程 (\ref{eq:RHP-2}) 改写为
\begin{equation}
  P^{-1}_{+} - P_{-} = (I - G)P^{-1}_{+} := \hat{C} P^{-1}_{+}
\end{equation}
由 Plemelj 公式可得
\begin{equation}
  P_{+} = I + \frac{1}{2 \pi \rmi} \int_{-\infty}^{\infty} \frac{\hat{C}(x,t,z) P_{+}^{-1}(x,t,z)}{s - z} \rmd s, \quad z \in \mathbb{C}_{+}
\end{equation}

Case2. 如果条件 (\ref{eq:RHP-regular}) 不满足, 则称 RH 问题为非正则的, 假设 $ \det P_{\pm} $ 在某些离散的点处为零, 由谱函数对称性 (\ref{eq:S-symmetry}) 有
\begin{equation}
  \det P_{+}(z) = S_{11}(z) = S_{22}^{*}(z^{*}) = \det P_{-}^{*}(z^{*}) = \det P_{-}(z^{*})
\end{equation}
故
\begin{equation}
  \det P_{+}(z) = 0 \iff \det P_{-}(Z^{*}) = 0,
\end{equation}
因此 $ \det P_{+}(z) $ 与 $ \det P_{-}(z) $ 有相同的零点个数, 且彼此共轭. 即若设 $ z_{j} (j = 1,2, \dots , N) $ 为 $ \det P_{+}(z) $ 在 $ \mathbb{C}_{+} $ 上的单零点, 则 $ z^{*}_{j} (j = 1,2, \dots , N) $ 为 $ \det P_{-}(z) $ 在 $ \mathbb{C}_{-} $ 上的单零点. 

由于 $ \det P_{+}(z_{j}) = \det P_{-}(z^{*}_{j}) = 0 $, 假设 $ w_{j}, w_{j}^{*} $ 分别为下列线性方程组的解
\begin{equation}
  P_{+}(z_{j}) w_{j}(z_{j}) = 0, \quad w_{j}^{*}(z_{j}^{*}) P_{-}(z_{j}^{*})  = 0 \label{eq:RHP-singular}
\end{equation}
对上式取共轭转置, 则有 $ w_{j}^{\dagger}(z_{j}) P_{+}^{\dagger}(z_{j})  = 0 $, 再利用对称性 (\ref{eq:P-symmetry}) 有
\begin{equation}
  w_{j}^{\dagger}(z_{j}) P_{-}(z_{j}^{*})  = 0 \label{eq:RHP-singular-Hermitian}
\end{equation}
比较可得 $ w_{j}^{\dagger}(z) = w_{j}^{*}(z^{*}) $.

非正则 RH 问题 (\ref{eq:RHP-1}) - (\ref{eq:RHP-3}) 的解可由下面定理给出
\begin{theorem}{Zakharov, Shabat, 1979}
  带有零点结构 (\ref{eq:RHP-singular}) 的非正则 RH 问题 (\ref{eq:RHP-1}) - (\ref{eq:RHP-3}) 可分解为
  \begin{equation}
    P_{+}(z) = \hat{P_{+}}(z) \Gamma(z) , \quad P_{-}(z) = \Gamma^{-1}(z) \hat{P_{-}}(z) \label{eq:RHP-decomposition}
  \end{equation}
  其中 
  \begin{equation}
    \Gamma(z) = I + \sum_{k,j = 1}^{N} \frac{w_{k}(M^{-1})_{kj} w_{j}^{*}}{z - z_{j}^{*}}, \quad \Gamma(z)^{-} = I - \sum_{k,j = 1}^{N} \frac{w_{k}(M^{-1})_{kj} w_{j}^{*}}{z - z_{k}} \label{eq:Gamma}
  \end{equation}
  这里 $ M $ 为 $ N \times N $ 矩阵, 其 $ (k.j) $ 元素由下式给定 
  \begin{equation}
    M_{kj} = \frac{w_{k}^{*}w_{j}}{z_{k}^{*} - z_{j}}, \quad k,j = 1,2, \dots, N, \quad \det \Gamma(z) = \prod_{k=1}^{N} \frac{z - z_{k}}{z - z_{k}^{*}}
  \end{equation}
  而 $ \hat{P_{\pm}} $ 为正则 RH 问题的唯一解, 且 
  \begin{enumerate}
    \item $ \hat{P_{\pm}} $ 在 $ \mathbb{C_{+}} $ 上解析,
    \item $ \hat{P_{-}} \hat{P_{+}} = \Gamma(z) G \Gamma^{-1}(z), z \in \mathbb{R} $
    \item $ \hat{P_{\pm}}(z) \to I, \quad z \to \infty $,
  \end{enumerate}
\end{theorem}
\begin{proof}
  非正则 RH 问题  (\ref{eq:RHP-1}) - (\ref{eq:RHP-3}) 是由于 $ 2N $ 个离散谱上的 $ \det P_{+}(z_{j}) = \det P_{-}(Z_{j^{*}}) = 0 (j = 1,2, \dots N) $ 造成的. 所以主要任务是消除这些零点结构, 并且去除 $ P_{+}(z_{j}), P_{-}(z_{j}^{*}) $ 在 $ z_{j}, z_{j}^{*} $ 上的零点结构, 为此需定义单极点矩阵
  \begin{equation}
    \Gamma_{1}(z) = I + \frac{z^{*}_{1} - z_{1}}{z - z_{1}^{*}} \cdot \frac{w_{1}w^{*}_{1}}{w^{*}_{1} w_{1}}
  \end{equation}
  其具有如下性质 
  \begin{equation}
    \begin{aligned}
      F_{1}^{-1}(z) &= I - \frac{z_{1}^{*} - z_{1}}{z - z_{1}} \cdot \frac{w_{1}w^{*}_{1}}{w^{*}_{1} w_{1}} \\
      \det \Gamma_{1}(z) &= \frac{z - z_{1}}{z - z_{1}^{*}}, \quad \det \Gamma_{1}^{-1}(z) = \frac{z - z_{1}^{*}}{z - z_{1}} \\
      \Gamma_{1}(z_{1}) w_{1} &= w_{1} + \frac{z^{*}_{1} - z_{1}}{z_{1} - z_{1}^{*}} \cdot \frac{w_{1}w_{1}^{*}}{w^{*}_{1}w_{1}} \cdot w_{1} = \frac{z^{*} - z_{1}}{z - z^{*}_{1}} \frac{w_{1}(w_{1}^{*}w_{1})}{w^{*}w_{1}} = w_{1} - w_{1} = 0 \\
      w_{1}^{*}\Gamma_{1}^{-1}(z^{*}_{1}) &= w_{1}^{*} - \frac{z_{1}^{*} - z_{1}}{z^{*} - z_{1}} \cdot \frac{w_{1}^{*}w_{1}}{w_{1}^{*}w_{1}} \cdot w_{1}^{*} = w_{1}^{*} - w_{1}^{*} = 0
    \end{aligned}
  \end{equation}
  令 $ x_{j} = \frac{w_{1}w_{1}^{*}}{w_{1}^{*}w_{1}} $, 则 $ x_{j} $ 为投影算子, 即 $ x_{j}^{2} = x_{j} $, 由上定义知 $ x_{j} $ 为一阶矩阵, 故与矩阵 $ \text{diga}\{1,0\} $ 相似, 既有可逆阵 $ T_{j} $ 使得 $ T_{j}^{-1} x_{j} T_{j} = \textrm{diag}\{1,0\} $, 从而有
  \begin{equation}
      \Gamma_{1}(z) = \det \left(I + \frac{z_{1}^{*} - z_{1}}{z - z_{1}^{*}} T_{j}^{-1}X_{j}T_{j}\right) = \begin{vmatrix}
      1 + \frac{z_{1}^{*} - z_{1}}{z - z_{1}^{*}} & 0 \\ 0 & 1
      \end{vmatrix} = \frac{z - z_{1}}{z - z^{*}_{1}}
  \end{equation}
  定义矩阵函数 $ R_{1}^{+}(z) = P_{+}(z) \Gamma_{+}^{-1}(z), R_{1}^{-1}= \Gamma_{1}(z) P_{-}(z) $, 由 (\ref{eq:RHP-singular}) - (\ref{eq:RHP-singular-Hermitian}) 知
  \begin{equation}
    \begin{aligned}
      \Res_{z = z_1} R_{1}^{+}(z) &= \Res_{z = z_{1}} \left(P_{+}(z) - P_{+}(z) \frac{z_{1}^{*} - z_{1}}{z - z_{1}} \frac{w_{1}w_{1}^{*}}{w_{1}^{*}w_{1}}\right) = - \frac{z_{1}^{*}-z_{1} }{w_{1}w_{1}^{*}}(P_{+}(z) w_{1}) w_{1}^{*} = 0 \\
      \Res_{z = z_{1}^{*}} R_{1}^{-1}(z) &= \Res_{z = z_{!}^{*}} \left(P_{1}(z) + \frac{z_{1}^{*} - z_{1}}{z - z_{1}} \frac{w_{1}w_{1}^{*}}{w_{1}^{*}w_{1}} P_{-}(z)\right) = - \frac{z_{1}^{*} - z_{1}}{w_{1}^{*}w_{1}} w_{1} (w_{1}^{*} P_{-}(z_{1}^{*})) = 0
    \end{aligned}
  \end{equation}
  因此 $ R_{1}^{+}(z), R_{1}^{-}(z) $ 分别在 $ \mathbb{C_{+}}, \mathbb{C_{-}} $ 上解析, 且
  \begin{equation}
    \begin{aligned}
      \det R_{1}^{+}(z_{1}) &= \lim_{z \to z_{1}} \left[ \det P_{+}(z) \cdot \det \Gamma_{1}^{-1}(z)\right] = \lim_{z \to z_{1}} S_{11}(z) \frac{z - z_{1}^{*}}{z-z_{1}} \\
      &= \lim_{z \to z_{1}} \frac{S_{11}(z) - S_{11}(z_{1})}{z - z_{1}} (z - z^{*}) \Leftarrow (S_{11}(z) = 0)\\
       &= \lim_{z \to z_{1}} S_{11}'(z) (z - z_{1}^{*}) = 0 \\
      \det R_{1}^{-1}(z_{1}^{*}) &= \lim_{z \to z_{1}^{*}} \left[ \det \Gamma_{1}(z) \cdot \det P_{-}(z)\right] = \lim_{z \to z_{1}^{*}} \frac{z - z_{1}}{z - z_{1}^{*}} S_{22}(z) = \lim_{z \to z_{1}^{*}} S_{22}'(z) (z - z_{1}) = 0
    \end{aligned}
  \end{equation}
  这说明 $ R_{1}^{+}(z), R_{1}^{-}(z) $ 分别在 $ z = z_{1}, z= z_{1}^{*} $ 不再具有零点结构, 然后去除其在 $ z_{2}, z_{2}^{*} $ 上的零点结构, 由于
  \begin{equation}
    \begin{aligned}
      \det R_{1}^{+}(z_{2}) &= \det P_{+}(z_{2}) \det \Gamma_{1}^{-1}(z_{2}) = S_{11}(z_{2}) \frac{z_{2} - z_{1}^{*}}{z_{2} - z_{1}} = 0 \\
      \det R_{1}^{-1}(z_{2}^{*}) &= \det \Gamma_{1}(z_{2}^{*}) \det P_{-}(z_{2}^{*}) = \frac{z_{2}^{*} - z_{1}}{z_{2}^{*} - z_{1}} S_{22}(z_{2}^{*}) = 0
    \end{aligned}
  \end{equation}
  因此, 下列齐次线性方程组有非零解, 即存在 $ v_{2}(z_{2}), v_{2}^{*}(z_{2}^{*}) $, 使得
  \begin{equation}
    R_{1}^{+}(z_{2}) v(z_{2}) = P_{+}(z_{2}) \Gamma_{1}^{-1}(z_{2})v_{2}(z_{2}) = 0, \quad v_{2}^{*}(z_{2}^{*}) R_{1}^{-1}(z_{2}^{*}) = v_{2}^{*}(z_{2}^{*}) \Gamma_{1}(z_{2}^{*}) P_{-}(z_{2}^{*}) = 0
  \end{equation}
  与 (\ref{eq:RHP-singular}) 比较可得
  \begin{equation}
    w_{2}(z_{2}) = \Gamma_{1}^{-1}(z_{2}) v_{2}(z_{2}), \quad w_{2}^{*}(z_{2}^{*}) = v_{2}^{*}(z_{2}^{*}) \Gamma_{1}(z_{2}^{*})
  \end{equation}
  为去除 $ R_{1}^{+}(z_{2}), R_{1}^{-}(z_{2}^{*}) $ 在 $ z_{2}, z_{2}^{*} $ 上的零点结构, 令
  \begin{equation}
    \begin{aligned}
    \Gamma_{2}(z) &= I + \frac{z_{2}^{*} - z_{2}}{z - z_{2}^{*}}\cdot \frac{v_{2}v_{2}^{*}}{v_{2}^{*}v_{2}}, \quad \det \Gamma_{2}(z) = \frac{z - z_{2}}{z - z_{2}^{*}} \\
    \Gamma_{2}^{-1}(z) &= I - \frac{z_{2}^{*} - z_{2}}{z - z_{2}} \cdot \frac{v_{2}v_{2}^{*}}{v_{2}^{*}v_{2}}, \quad  \det \Gamma_{2}^{-1} = \frac{z - z_{2}^{*}}{z - z_{2}}\\
    R_{2}^{+}(z) &= P_{1}^{+}(z) \Gamma_{2}^{-1}(z) = P_{1}^{+}(z) \Gamma_{1}^{-1}(z) \Gamma_{2}^{-1}(z) \\
    R_{2}^{-1}(z) &= \Gamma_{2}(z) P_{1}^{-1}(z) = \Gamma_{2}(z) \Gamma_{1}(z) P_{1}^{-1}(z)
  \end{aligned}
  \end{equation}
  则
  \begin{equation}
    \Res_{z = z_{2}} R_{2}^{+}(z) = \Res_{z = z_{2}} \left[ \left(P_{+}(z) - P_{+}(z) \frac{z_{1}^{*} - z_{1}}{z - z_{1}} \cdot \frac{w_{1}w_{1}^{*}}{w_{2}^{*}w_{1}}\right) \left( I -\frac{z_{1}^{*} - z_{1}}{z - z_{1}} \cdot \frac{w_{1}w_{1}^{*}}{v_{2}^{*}v_{1}} \right)\right] = - \frac{z_{2}^{*} - z_{1}R_{1}(z_{2})v_{2}v_{2}^{*}}{v_{2}^{*}v_{2}}
  \end{equation}
  同理 $ \Res_{z = z_{2}^{*}} R_{2}^{-1}(z) = 0 $, 由此可得 $ R_{2}^{+}(z), R_{2}^{-1}(z) $ 分别在在 $ z=z_{1}, z_{2}, z=z_{1}^{*}, z_{2}^{*} $ 上无零点结构, 则
  \begin{equation}
    \begin{aligned}
    \det R_{2}^{+}(z_{j}) &= \lim_{z \to z_{j}} \left[ \det P_{+}(z) \det \Gamma_{1}^{-1}(z) \det \Gamma_{2}^{-1}(z)\right] \\
    &= \lim_{z \to z_{j}} S_{11}(z) \frac{z - z_{1}^{*}}{z - z_{1}} \cdot \frac{z - z_{2}^{*}}{z - z_{2}} \\
    &= s'_{11}(z_{j}) \frac{(z_{j} - z_{1}^{*})(z_{j} - z_{2}^{*})}{z_{j} - z_{k}} \neq 0 \\
    \det R_{2}^{-}(z_{j}^{*}) &\neq 0 \quad (k = 1, 2, k \neq z)
    \end{aligned}
  \end{equation}
  更一般的, 可得 
  \begin{equation}
    \begin{aligned}
      w_{j} &= \Gamma_{1}^{-1}(z_{j}) \cdots \Gamma_{j-1}^{-1}(z_{j}) \cdot v_{j}(z_{j}) \\
      w_{j}^{*} &= v_{j}^{*}(z_{j}^{*}) \cdot \Gamma_{1}(z_{j}^{*}) \cdots \Gamma_{j-1}(z_{j}^{*}) \\
      R_{j}^{+}(z) &= P_{+}(z) \cdot \Gamma_{1}^{-1}(z) \cdots \Gamma_{j}^{-1}(z) \\
      R_{j}^{-1}(z) &= \Gamma_{j}(z) \cdots \Gamma_{1}(z) \cdot P_{-}(z)
    \end{aligned}
  \end{equation}
  其中 
  \begin{equation}
    \Gamma_{j}(z) = I + \frac{z_{j}^{*} - z_{j}}{z - z_{j}^{*}} \cdot \frac{v_{j}v_{j}^{*}}{v_{j}^{*}v_{j}}, \quad \Gamma_{j}^{-1}(z) = I - \frac{z_{j}^{*} - z_{j}}{z - z_{j}} \cdot \frac{v_{j}v_{j}^{*}}{v_{j}^{*}v_{j}} \label{eq:Gamma-j}
  \end{equation}
  则 $ R_{j}^{+}(z), R_{j}^{-1}(z) $ 在 $ \mathbb{C_{+}}, \mathbb{C_{-}} $ 上解析, 且在 $ z_{k}, z_{k}^{*} (k = 1,2, \ldots j) $ 上无零点结构, 即
  \begin{equation}
    \begin{aligned}
      \det R_{j}^{+}(z_{k}) = s'_{11}(z_{k}) \frac{\prod_{l = 1}^{j}(z_{k} - z_{l}^{*})}{\prod_{l = 1, l \neq k }^{j}(z_{k} - z_{l})} \neq 0 \\
      \det R_{j}^{-1}(z_{k}^{*}) = s'_{22}(z_{k}^{*}) \frac{\prod_{l = 1}^{j}(z_{k}^{*} - z_{l})}{\prod_{l = 1, l \neq k }^{j}(z_{k}^{*} - z_{l})} \neq 0
    \end{aligned}  \quad (k = 1, 2, \dots, j)
  \end{equation}
  而
  \begin{equation}
    \begin{aligned}
      \det R_{j}^{+}(z_{j+1}) &= S_{11}(z_{j+1}) \prod_{j = 1}^{j} \frac{z_{j+1} - z_{k}^{*}}{z_{j+1} - z_{k}} = 0 \\
      \det R_{j}^{-1}(z_{j+1}^{*}) &= S_{22}(z_{j+1}^{*}) \prod_{j = 1}^{j} \frac{z_{j+1}^{*} - z_{k}}{z_{j+1}^{*} - z_{k}} = 0
    \end{aligned}
  \end{equation}
  最后令 
  \begin{equation}
    \begin{aligned}
      \Gamma_{z} = \Gamma_{N}(z) \cdots \Gamma_{1}(z),& \quad \Gamma_{z}^{-1} = \Gamma_{1}^{-1}(z) \cdots \Gamma_{N}^{-1}(z) \label{eq:Gamma-composition} \\ 
      \hat{P_{+}}(z) = P_{+}(z) \Gamma_{z}^{-1}(z),& \quad \hat{P_{-}}(z) = \Gamma_{z}(z) P_{-}(z)
    \end{aligned}
  \end{equation}
  则 $ \hat{P_{+}}(z), \hat{P_{-}}(z) $ 在 $ \mathbb{C_{+}}, \mathbb{C_{-}} $ 上解析, 且在 $ z_{j}, z_{j}^{*} (j = 1,2, \dots, N) $ 上无零点结构 (实际上在解析区域内无零点结构), 即
  \begin{equation}
    \det \hat{P_{+}}(z_{k}) \neq 0, \quad \det \hat{P_{-}}(z_{k}^{*}) \neq 0 \quad (k = 1,2, \dots, N) 
  \end{equation}
  得到分解 (\ref{eq:RHP-decomposition}). 下证 (\ref{eq:Gamma}). 注意到 $ \Gamma(z), \Gamma^{-1}(z) $ 分别为具有单极点 $ z_{j}, z_{j}^{*} (1 \leq j \leq N)$ 的亚纯函数, 以及分解 (\ref{eq:Gamma-j}), (\ref{eq:Gamma-composition}), 寻找的向量使得
  \begin{equation}
    \Gamma(Z) = i + \sum_{j=1}^{N} \frac{\xi_{j}w_{j}^{*}}{z-z_{j}^{*}}, \quad \Gamma^{-1}(z) = I - \sum_{j=1}^{N} \frac{w_{j}^{*}\xi_{j}}{z - z_{j}} \label{eq:Gamma-expRession}
  \end{equation} 
  注意到 $ \Gamma(z) \Gamma^{-1}(z) = I $ 对任意 $ z $ 都成立, 当然也在 $ z = z_{k} $ 处成立, 即 $ \Gamma(z) \Gamma^{-1}(z) = I $ 在 $ z = z_{k} $ 处正则, 为保证 $ \Gamma(z) \Gamma^{-1}(z) = I $ 在 $ z = z_{k} $ 处成立, 只需让其留数为 $ 0 $, 因此利用 (\ref{eq:Gamma-expRession}) 可得
  \begin{equation}
    \begin{aligned}
      0 &= \Res_{z = z_{k}} [\Gamma(z)\Gamma^{-1}(z)] = \Res_{z = z_{k}} \left( \Gamma(z) - \sum_{j = 1}^{N} \Gamma(z)\frac{w_{j}\xi_{j}^{*}}{z- z_{j}} \right) = - \Gamma(z_{k}) w_{k} \xi_{k}^{*} \\
      & - \left( I + \sum_{j = 1}^{N} \frac{\xi_{j}w_{j}^{*}}{z_{k} - z_{j}^{*}} \right) w_{k} \xi_{k}^{*} = \left( - w_{k} + \sum_{j = 1}^{N} \frac{w_{j}^{*}w_{k}}{z_{j}^{*} - z_{k}} \xi_{j}^{*} \right) \xi_{k}^{*}
    \end{aligned}
  \end{equation}
  对上式两边同时作用 $ \xi_{k} $, 则有 
  \begin{equation}
    \left( - w_{k} + \sum_{j = 1}^{N} \frac{w_{j}^{*}w_{k}}{z_{j}^{*} - z_{k}} \xi_{j} \right) | \xi_{k} |^{2} = 0 \implies  - w_{k} + \sum_{j = 1}^{N} \frac{w_{j}^{*}w_{k}}{z_{j}^{*} - z_{k}} \xi_{j} = 0 \quad (1 \leq k \leq N)
  \end{equation}
  将上式改写为 $ \xi_{1}, \xi_{2}, \dots, \xi_{N} $ 的分块矩阵形式的线性方程组, 则有
  \begin{equation}
    (\xi_{1}, \xi_{2}, \dots, \xi_{N}) M =  (w_{1}^{*}, w_{2}^{*}, \dots, w_{N}^{*})
  \end{equation}
  其中 $ M = (M_{kj})_{N \times N} $, 其中 $ M_{kj} = \frac{w_{k}^{*}w_{j}}{z_{k}^{*} - z_{j}} $ 则
  \begin{equation}
    \begin{cases}
      \frac{w_{1}^{*}w_{1}}{z_{1}^{*}- z_{1}} \xi_{1} + \frac{w_{2}^{*}w_{1}}{z_{2}^{*} - z_{1}} \xi_{2} + \dots + \frac{w_{N}^{*}w_{1}}{z_{N}^{*} - z_{1}} \xi_{N} = w_{1} \\
      \frac{w_{1}^{*}w_{2}}{z_{1}^{*}- z_{2}} \xi_{1} + \frac{w_{2}^{*}w_{2}}{z_{2}^{*} - z_{2}} \xi_{2} + \dots + \frac{w_{N}^{*}w_{2}}{z_{N}^{*} - z_{2}} \xi_{N} = w_{2} \\
      \vdots \\
      \frac{w_{1}^{*}w_{N}}{z_{1}^{*}- z_{N}} \xi_{1} + \frac{w_{2}^{*}w_{N}}{z_{2}^{*} - z_{N}} \xi_{2} + \dots + \frac{w_{N}^{*}w_{N}}{z_{N}^{*} - z_{N}} \xi_{N} = w_{N}
    \end{cases} 
    \implies M = \begin{pmatrix}
      \frac{w_{1}^{*}w_{1}}{z_{1}^{*} - z_{1}} & \frac{w_{1}^{*}w_{2}}{z_{1}^{*} - z_{2}} & \dots & \frac{w_{1}^{*}w_{N}}{z_{1}^{*} - z_{N}} \\
      \frac{w_{2}^{*}w_{1}}{z_{2}^{*} - z_{1}} & \frac{w_{2}^{*}w_{2}}{z_{2}^{*} - z_{2}} & \dots & \frac{w_{2}^{*}w_{N}}{z_{2}^{*} - z_{2}} \\
      \vdots & \vdots & \ddots & \vdots \\
      \frac{w_{N}^{*}w_{1}}{z_{N}^{*} - z_{1}} & \frac{w_{N}^{*}w_{2}}{z_{N}^{*} - z_{2}} & \dots & \frac{w_{N}^{*}w_{N}}{z_{N}^{*} - z_{N}}
    \end{pmatrix}
  \end{equation}
  可得 $ \xi_{j} = \sum_{k =1}^{N}(M^{-1})_{kj} w_{k} $. 最后将其带入 (\ref{eq:Gamma-expRession})得到 (\ref{eq:Gamma})
\end{proof}

\section{NLS 方程的 N 孤子解}
\subsection{矩阵向量解的时空演化}
对方程 (\ref{eq:RHP-singular}) 第一个式子两边分别对 $ x, t $ 求导, 可得
\begin{equation}
  P_{+, x} w_{j} + P_{+} w_{j, x} = 0, \quad P_{+, t} w_{j} + P_{+} w_{j, t} = 0 \label{eq:Focus-NLS-Lax-x-t-Asymptotic}
\end{equation}
利用 $ P_{+} $ 的定义与 Lax 对(\ref{eq:Focus-NLS-Lax-x-Asymptotic}), (\ref{eq:Focus-NLS-Lax-t-Asymptotic}) 可得
\begin{equation}
  \begin{aligned}
  P_{+,x} &= \mu_{1,x} H_{1} + \mu_{2,x} H_{2} = \left( \rmi z_{j} [\sigma_{3},\mu_{1}] + P\mu_{1} \right) H_{1} + \left( - \rmi z_{j} [\sigma_{3}, \mu_{2}] + P \mu_{2} \right) H_{2} \\
    & - \rmi z_{j} \left( \mu_{1} \sigma_{3} H_{1} - \mu_{2} \sigma_{3} H_{2} + \sigma_{3} \mu_{1} H_{1} + \sigma_{3} \mu_{2} H_{2} \right) + P \mu_{1} H_{1} + P \mu_{2} H_{2} \\ 
    &= - \rmi z [\sigma_{3}, p_{+}] + P P_{+}
  \end{aligned} \label{eq:P-x}
\end{equation}
同理有 
\begin{equation}
 P_{+,t} = - \rmi z_{j}^{2} [\sigma_{3}, P_{+}] + Q P_{+} \label{eq:P-t}
\end{equation} 将 (\ref{eq:P-x}), (\ref{eq:P-t}) 代入 (\ref{eq:Focus-NLS-Lax-x-t-Asymptotic}) 且由于 $ P_{+}w_{j} = 0, w_{j}P = 0 $, 有
\begin{equation}
  ( - \rmi z_{j} [\sigma_{3}, P_{+}] + P P_{+})w_{j} + P_{+}w_{j,x} = 0 \implies \rmi z_{j} P_{+} \sigma_{3} w_{j} + P_{+} w_{j,x} = 0 \implies P_{+}(w_{j,x} + \rmi z_{j} \sigma_{3}w_{j}) = 0
\end{equation}
同理有 $ P_{+}(w_{j,t} + \rmi z_{j}^{2} \sigma_{3} w_{j}) = 0 $. 
\begin{equation}
  \begin{cases}
    w_{j,x} + \rmi z_{j} \sigma_{3} w_{j} = 0 \\
    w_{j,t} + \rmi z_{j}^{2} \sigma_{3} w_{j} = 0
  \end{cases} \implies w_{j} = e^{- \rmi \theta(z_{j}) \sigma_{3}} w_{j,0}, \quad (j = 1,2, \dots, N)
\end{equation}
其中 $ w_{j,0} $ 为 2 维常向量, 从而 $ w_{j}^{*} = w_{j,0}^{\dagger} e^{\rmi \theta(z_{j}^{*}) \sigma_{3}} $. 

\subsection{N 维孤子解公式}
已知 $ P_{-} P_{+} = G \implies P_{+}^{-1} - P_{-} = \hat{G}P_{+} $(其中 $ I - G = \hat{G} $), 且 $ \hat{P_{-}}(z) \hat{P_{+}}(z)  = \Gamma(z) G \Gamma^{-1}(z) (z \in \mathbb{R})$, 可得 
\begin{equation}
  \begin{aligned}
  \hat{P_{+}^{-1}}- \hat{P_{-}} &= (I - \hat{P_{-}} \hat{P_{+}}) \hat{P_{+}^{-1}} = \left(I - \Gamma (z) G \Gamma^{-1}(z) \right) \hat{P_{+}^{-1}} = (\Gamma(z)\Gamma^{-1}(z) - \Gamma(z) G \Gamma^{-1}(z)) \hat{P_{+}^{-1}} \\
    &= \Gamma(z) (I-G) \Gamma^{-1}(z)\hat{P_{+}^{-1}} = \Gamma(z) \hat{G} \Gamma^{-1}(z) \hat{P_{+}^{-1}} 
  \end{aligned}
\end{equation}
由 Taylor 公式 $ \frac{1}{s - z} = - \frac{1}{z} (\frac{1}{1- s/z}) = - \frac{1}{z} \left(1 + \frac{s}{z} + \cdots\right) $, 故 Plemelj 公式可写为
\begin{equation}
  \begin{aligned}
    \hat{P_{+}^{-1}} &= I + \frac{1}{2 \pi \rmi } \int_{-\infty}^{+\infty} \frac{\Gamma(z) \hat{G} \Gamma^{-1}(z) \hat{P_{+}^{-1}} }{s - z} \rmd s \\
    &= I - \frac{1}{2 \pi \rmi z} \int_{-\infty}^{+\infty} \left( 1 + \frac{s}{z} + (\frac{s}{z})^{2} + \cdots \right) \Gamma(z) \hat{G} \Gamma^{-1}(z) \hat{P_{+}^{-1}} \rmd s \\
    &= I - \frac{1}{2 \pi \rmi z} \int_{-\infty}^{+\infty}  \Gamma(z) \hat{G} \Gamma^{-1}(z) \hat{P_{+}^{-1}} \rmd s + O(z^{-2}) \\
  \end{aligned}
\end{equation}
由 $ (I - A)^{-1} = I + A + A^{2} + \cdots, (P^{-1} = I + \frac{A}{z} + \dots, P = \frac{B}{z} + \dots) $, 可得
\begin{equation}
  \hat{P_{+}} = I + \frac{1}{2 \pi \rmi z} \int_{-\infty}^{+\infty} \Gamma(z) \hat{G} \Gamma^{-1}(z) \hat{P_{+}} \rmd s + O(z^{-2})
\end{equation}
再由 
\begin{equation}
  \Gamma(z) = i + \sum_{k,j = 1}^{N} \frac{w_{k}(M^{-1})_{kj}w_{j}^{*}}{z-z_{j}^{*}} \implies \Gamma(z) = I + \frac{1}{z} \sum_{k,j = 1}^{N} w_{k}(M^{-1})_{kj}w_{j}^{*} + O(z^{-2}) \label{eq:Gamma-Asymptotic}
\end{equation}
将上渐进式带入 (\ref{eq:RHP-decomposition}), 比较 $ z^{-1} $ 的次数可得
\begin{equation}
  P_{+}^{(1)} = \frac{1}{2 \pi \rmi} \int_{-\infty}^{+\infty} \Gamma(z) \hat{G} \Gamma^{-1}(z) \hat{P_{+}^{-1}} \rmd s = \sum_{k,j = 1}^{N} w_{k} (M^{-1})_{kj} w_{j}^{*} \label{eq:P-+-1}
\end{equation}
特别的, 当散射数据 $ S_{12} = S_{21} = 0 $ 时, 有 $ \hat{G} = I - G = 0 $. 故上式(\ref{eq:P-+-1}) 可简化为
\begin{equation}
  P_{+}^{-1} = \sum_{k,j = 1}^{N} w_{k}(M^{-1})_{kj} w_{j}^{*}
\end{equation}
不妨取 $ \lambda_{j} = - \rmi (z_{j}x + 2 z_{j}^{2} t) $, 并取 $ w_{j,0} = (c_{j},1)^{T} $, 则有
\begin{equation}
  w_{j} = \begin{pmatrix}
    e^{\lambda_{j}} & \\ & e^{-\lambda_{j}}
  \end{pmatrix} \begin{pmatrix}
    c_{j} \\ 1
  \end{pmatrix} = \begin{pmatrix}
    c_{j} e^{\lambda_{j}} \\ e^{-\lambda_{j}}
  \end{pmatrix}
\end{equation}
从而 $ w_{j}^{*} = w_{j,0}^{*} \cdot e^\lambda_{j} \sigma_{3} = (c_{j}^{*}e^{\lambda_{j}^{*}}, e^{-\lambda_{j}})$. 故
\begin{equation}
  M_{k.j} = \frac{w_{k}^{*}w_{j}}{z_{k}^{*}- z_{j}} = \frac{1}{z_{k}^{*} - z_{j}} (c_{k}^{*}c_{j} e^{\lambda_{k}^{*} + \lambda_{j}} + e^{-\lambda_{k} - \lambda_{j}})
\end{equation}
再由 (\ref{eq:Gamma-Asymptotic}) 
\begin{equation}
  \begin{aligned}
    q(x,t) &= 2 \rmi \lim_{z \to \infty} (zP_{+})_{12} = 2 \rmi (P_{+}^{(1)})_{12}\\
     &= 2 \rmi \sum_{k,j}^{N}(w_{k}w_{j}^{*})_{12}(M^{-1})_{kj} = 2 \rmi \sum_{k,j}^{N} c_{k}e^{\lambda_{k} - \lambda_{j}^{*}}(M^{-1})_{kj} \label{eq:NLS-N-soliton}
  \end{aligned}
\end{equation}
令
\begin{equation}
  R = \begin{pmatrix}
    0 & c_{1}e^{\lambda_{1}} & \cdots & c_{N}e^{\lambda_{N}} \\
    e^{-\lambda_{1}^{*}} & M_{11} & \cdots & M_{1N} \\
    \vdots & \vdots & \ddots & \vdots \\
    e^{-\lambda_{N}^{*}} &  M_{N1} & \cdots & M_{NN}
  \end{pmatrix}
\end{equation}
则 
\begin{equation}
  \begin{aligned}
    \det R &= \sum_{k=1}^{N}(-1)^{k+2} c_{k} e^{\lambda_{k}} \det \begin{pmatrix}
      e^{-\lambda_{1}^{*}} & M_{11} & \cdots & M_{1,k-1} & M_{1,k+1} & \cdots & M_{1N} \\
      \vdots & \vdots & \ddots & \vdots & \vdots & \ddots & \vdots \\
      e^{-\lambda_{N}^{*}} & M_{N1} & \cdots & M_{N,k-1} & M_{N,k+1} & \cdots & M_{NN}
    \end{pmatrix} \\
    &= \sum_{k=1}^{N} (-1)^{k+2} c_{k} e^{\lambda_{k}} \sum_{j=1}^{N} e^{-\lambda_{j}^{*}} \det \underline{\begin{pmatrix}
      M_{11} & \cdots & M_{1,k-1} & M_{1,k+1} & \cdots & M_{1N} \\
      \vdots & \ddots & \vdots & \vdots & \ddots & \vdots \\
      M_{N1} & \cdots & M_{N,k-1} & M_{N,k+1} & \cdots & M_{NN}
    \end{pmatrix}} := \Delta \\
    &= \sum_{k=1}^{N}(-1)^{k+j+3}c_{k} e^{\lambda_{k} - \lambda_{j}^{*}} \det(\Delta) \\
    %\overset{\mathclap{\text{这是较长的文本,需要等号自适应}}}{=} \\
    &= -c_{k} e^{\lambda_{k} - \lambda_{j}^{*} (M^{*})_{jk}} \qquad \Leftarrow \left((M^{*})_{jk} - (-1)^{k+j} \det(\Delta) \right) \\
    &= -\det M \sum c_{k}e^{\lambda_{k} - \lambda_{j}^{*}} \qquad \Leftarrow \left( (M^{-1})_{jk} (M^{*}) = |M| M^{-1} \right)
  \end{aligned}
\end{equation}
因此 
\begin{equation}
  \sum_{k=1}^{N} c_{k} e^{\lambda_{k} - \lambda_{j}^{*}}(M^{-1})_{jk} = - \frac{\det R}{\det M} \label{eq:q}
\end{equation}
将 (\ref{eq:q}) 带入 (\ref{eq:NLS-N-soliton}) 可得 NLS 方程的 N 孤子解
\begin{equation}
  q = -2 \rmi \frac{\det R}{ \det M}
\end{equation}